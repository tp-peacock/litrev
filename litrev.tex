\documentclass[12pt,a4paper]{article}

\begin{document}
	
	\title{\textbf{Literature Review}}
	\author{\textbf{Thomas P. Peacock}}
	\date{\textbf{February 2018}}
	\maketitle
	
\section{Binding Affinity}

Ref for section:\\ \textbf{https://www.ncbi.nlm.nih.gov/pmc/articles/PMC4523921/} 
\\
\\
Different methods for predicting binding affinity (BA) have been used, varying in accuracy and computational cost. \textbf{Exact methods} such as \textbf{free energy perturbation} and \textbf{thermodynamics integration} can be very accurate, but have limited application due to their computational cost. Mostly for low throughput studies and for small drug binding or mutation REF. Are TCR complexes small enough for this approach? If not on a large scale, could it be used as a final accurate process once a good candidate has been selected? 
\\
\\
Methods based on empirical functions are much faster - \textbf{empirical, force-field-based potentials, statistical potentials, scoring functions used in docking}. Lots of references given by REF. The main weaknesses of these methods are that they usually neglect factors such as conformational changes upon binding, allosteric regulation, and solvent and co-factor effects, which may all contribute to the binding strength.
\\
\\
Allosteric regulation: regulation of an enzyme by binding an effector molecule at a site different to the enzyme's active site. The site which the effector binds is the allosteric site.

\end{document}