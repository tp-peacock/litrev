%\documentclass[fleqn,10pt]{SelfArx} % Document font size and equations flushed left
%
%\usepackage[english]{babel} % Specify a different language here - english by default
%
%\usepackage{lipsum} % Required to insert dummy text. To be removed otherwise
%
%\usepackage[backend=bibtex,style=nature]{biblatex}
%\usepackage{graphicx}
%\usepackage{subfig}
%\usepackage{booktabs}
%
%
%
%\begin{document}
%	
%
%	aamir paper \cite{aamir_2013}
%
%	\printbibliography
%
%\end{document}

%%%%%%%%%%%%%%%%%%%%%%%%%%%%%%%%%%%%%%%%%
% Stylish Article
% LaTeX Template
% Version 2.1 (1/10/15)
%
% This template has been downloaded from:
% http://www.LaTeXTemplates.com
%
% Original author:
% Mathias Legrand (legrand.mathias@gmail.com) 
% With extensive modifications by:
% Vel (vel@latextemplates.com)
%
% License:
% CC BY-NC-SA 3.0 (http://creativecommons.org/licenses/by-nc-sa/3.0/)
%
%%%%%%%%%%%%%%%%%%%%%%%%%%%%%%%%%%%%%%%%%

%----------------------------------------------------------------------------------------
%	PACKAGES AND OTHER DOCUMENT CONFIURATIONS
%----------------------------------------------------------------------------------------

\documentclass[12pt]{SelfArx} % Document font size and equations flushed left



\usepackage[english]{babel} % Specify a different language here - english by default

\usepackage[backend=bibtex,style=nature]{biblatex}
	\bibliography{bibliography}

\usepackage{lipsum} % Required to insert dummy text. To be removed otherwise
\usepackage{afterpage}
\usepackage[T1]{fontenc} 
\usepackage{csquotes}
\usepackage[font=small,labelfont=bf,
justification=justified,
format=plain]{caption} 
\usepackage{dblfloatfix}
\usepackage{subfig}



\providecommand{\e}[1]{\ensuremath{\times 10^{#1}}}

%----------------------------------------------------------------------------------------
%	COLUMNS
%----------------------------------------------------------------------------------------

%\setlength{\columnsep}{0.55cm} % Distance between the two columns of text
\setlength{\fboxrule}{0pt} % Width of the border around the abstract

%----------------------------------------------------------------------------------------
%	COLORS
%----------------------------------------------------------------------------------------

\definecolor{color1}{RGB}{0,0,90} % Color of the article title and sections
\definecolor{color2}{RGB}{255,255,255} % Color of the boxes behind the abstract and headings

%----------------------------------------------------------------------------------------
%	HYPERLINKS
%----------------------------------------------------------------------------------------


\usepackage{hyperref} % Required for hyperlinks
\hypersetup{hidelinks,colorlinks,breaklinks=true,urlcolor=color1,citecolor=color1,linkcolor=color1,bookmarksopen=false,pdftitle={Title},pdfauthor={Author}}

%----------------------------------------------------------------------------------------
%	ARTICLE INFORMATION
%----------------------------------------------------------------------------------------

\JournalInfo{Literature Review} % Journal information
\Archive{CoMPLEX/Division of Infection \& Immunity} % Additional notes (e.g. copyright, DOI, review/research article)

\PaperTitle{Literature Review} % Article title

\Authors{Thomas P. Peacock\textsuperscript{1 2}}
% Authors
\affiliation{\textsuperscript{1}\textit{CoMPLEX, University College London, London, United Kingdom}} % Author affiliation
\affiliation{\textsuperscript{2}\textit{Division of Infection and Immunity, University College London, London, United Kingdom}}
	
%\affiliation{\textsuperscript{2}\textit{}} % Author affiliation
%\affiliation{*\textbf{Corresponding author}:} % Corresponding author

\Keywords{} % Keywords - if you don't want any simply remove all the text between the curly brackets
\newcommand{\keywordname}{Keywords} % Defines the keywords heading name

\renewcommand{\abstractname}{}    % clear the title
\renewcommand{\abstractname}{\vspace{-\baselineskip}}


%----------------------------------------------------------------------------------------
%	ABSTRACT
%----------------------------------------------------------------------------------------
\Abstract{}	

%----------------------------------------------------------------------------------------
\hypersetup{pdftitle=Literature Review, pdfauthor=Thomas P. Peacock}
\begin{document}
	
	\flushbottom % Makes all text pages the same height
	
	\maketitle % Print the title and abstract box

	\thispagestyle{empty} % Removes page numbering from the first page

\section{Binding Affinity}

Different methods for predicting binding affinity (BA) have been used, varying in accuracy and computational cost. \textbf{Exact methods} such as \textbf{free energy perturbation} and \textbf{thermodynamics integration} can be very accurate, but have limited application due to their computational cost. Mostly for low throughput studies and for small drug binding or mutation \cite{Vangone2015}. Are TCR complexes small enough for this approach? If not on a large scale, could it be used as a final accurate process once a good candidate has been selected? 
\\
\\
Methods based on empirical functions are much faster - \textbf{empirical, force-field-based potentials, statistical potentials, scoring functions used in docking}. Lots of references given by \cite{Vangone2015}. The main weaknesses of these methods are that they usually neglect factors such as conformational changes upon binding, allosteric regulation, and solvent and co-factor effects, which may all contribute to the binding strength.

\subsubsection{Allosteric regulation}
Regulation of an enzyme by binding an effector molecule at a site different to the enzyme's active site. The site which the effector binds is the allosteric site.

	


	
	
	%----------------------------------------------------------------------------------------
	%	REFERENCE LIST
	%----------------------------------------------------------------------------------------
	\phantomsection

	\printbibliography	
	%----------------------------------------------------------------------------------------
	
\end{document}